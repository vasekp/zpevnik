\pisen{J. Nohavica}{Darmoděj}

\sloka
[Ami]Šel včera městem [Emi]muž a šel po hlavní [Ami]třídě, [Emi\ Ami]
šel včera městem [Emi]muž a já ho z okna [Ami]viděl. [Emi]
[C]Na flétnu chorál [G]hrál, znělo to jako [Ami]zvon
a byl v tom všechen [Emi]žal, ten krásný dlouhý [F]tón,
a já jsem náhle [F#\dim]věděl: Ano, to je [E7]on, to je [Ami]on.

\sloka
Vyběh’ jsem do ulic jen v noční košili,
v odpadcích z popelnic krysy se honily
a v teplých postelích lásky i nelásky
tiše se vrtěly rodinné obrázky,
a já chtěl odpověď na svoje otázky, otázky.

\napis{M.}
[*Ami Emi C G Ami F F#\dim E7 Ami Emi C G Ami F F#\dim E7]

\sloka
Dohnal jsem toho muže a chytl za kabát,
měl kabát z hadí kůže, šel z něho divný chlad,
a on se otočil a oči plné vran a jizvy u očí, celý byl pobodán
a já jsem náhle věděl kdo je onen pán, onen pán.

\sloka
Celý se strachem chvěl, když jsem tak k němu došel,
a v ústech flétnu měl od Hieronyma Bosche,
stál měsíc nad domy jak čírka ve vodě,
jak moje svědomí, když zvrací v záchodě,
a já jsem náhle věděl: to je Darmoděj, můj Darmoděj.

\ref
[Ami]Můj Darmo[Emi]děj, vaga[C]bund osu[G]dů a lásek,
[Ami]jenž prochá[F]zí všemi [F#mi]sny, ale [E7]dnům vyhýbá se,
[Ami]můj Darmo[Emi]děj, krásné [C]zlo, jed má [G]pod jazykem,
[Ami]když prodá[F]vá po do[F#mi]mech jehly [E7]se slovníkem.

\sloka
Šel včera městem muž, podomní obchodník,
šel, ale nejde už, krev skápla na chodník,
já jeho flétnu vzal a zněla jako zvon
a byl v tom všechen žal, ten krásný dlouhý tón,
a já jsem náhle věděl:
ano, já jsem on, já jsem on.

\ref
Váš Darmoděj, vagabund osudů a lásek,
jenž prochází všemi sny, ale dnům vyhýbá se,
váš Darmoděj krásné zlo jed mám pod jazykem
když prodávám po domech jehly se slovníkem.
