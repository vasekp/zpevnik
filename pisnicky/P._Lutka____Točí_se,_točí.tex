\pisen{P. Lutka}{Točí se, točí}

\sloka
[A]Točí se, točí [E]kola ve mlý[A]ně.
Točí se, točí, já [D]jsem v tom nevin[A]ně.
Točí se [E]kola, voda vzduchem [A]padá
a já tomu mlejnu [E]ukazuju [A]záda.

\sloka
V tom mlejně žila překrásná dívka
a já ji měl tuze a tuze moc rád.
Měla krásný oči a černý vlasy,
jenom jednu vadu, že nechtěla mi dát.

\sloka
Nechtěla mi dáti ani políbení,
nechtěla mi dáti ani hubičku.
Proto jsem ji musel, tu potvoru jednu,
proto jsem ji musel topit v rybníčku.

\sloka
Když se dcera domů dlouho nevracela,
její otec mlynář o ni starost měl.
Proto poslal mládka kouknout se, co dělá,
proto poslal mládka, aby za ní šel.

\sloka
Když tam mládek přišel, zrovna topila se,
její ručka bílá o pomoc prosila.
Pomoct jí však nemoh, kosa byla ostrá,
jeho mladý čelo brzo trefila.

\sloka
Když se mládek s dcerou dlouho nevraceli,
přišel se sám mlynář po dcerušce ptát.
Když ho kola mlýnský na prach rozemlely,
řekl jsem si, hochu, musíš na cestu se dát.

\sloka
= 1.
