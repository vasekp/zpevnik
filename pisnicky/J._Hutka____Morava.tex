\pisen{J. Hutka}{Morava}

\sloka
[Ami]Brněnská radnice stavěná do vejš[E]ky,
v ulicích práší se a nejvíc na Čes[Ami]ký [E7]
[Ami]Dětičky s jásotem z radnice máva[Dmi]jí,
[C]holubi nad měs[E]tem poletu[Ami]jí.

\ref
[G7]A Mora[C]va je krásná zem, na její [G7]slávu připijem,
só na ní hezká děvčata tvářičky [C]majó ze zlata.
Dobrý člo[Dmi]věk ještě ži[E]je: na Moravě víno pi[Ami]je.
Proto ať vzkvétá Mora[C]va a je[G7]jí slá[C]va. [E]

\sloka
V Špilberku na kopci v domečku kamenným
divěj se cizinci jakej’s byl na ženy.
Děti se poučí dobou fe-udální –
silný slabé mučí, to dnes už není.

\ref
A Morava…

\sloka
A brněnský kolo, docela dřevěný,
letos prohánělo motorky závodní.
Dětičky mávaly Frantovi Šťastnýmu,
na dráze spatřily samou šmouhu.

\ref
A Morava…

\sloka
Copak to vyrostlo na Královým poli,
už to snad uzrálo, že to nesklidili?
Dětičky mávají komínům továrním,
radostně vdechují ten jejich dým.

\ref
A Morava…

\zlom

\sloka
No a brněnskej drak v Nilu se narodil,
dětem nahání strach, má jméno Krokodil.
Visí na řetizku, v průvanu houpe se,
píšu do notysku, ať nekrmí se.

\ref
A Morava…

\napis{}
\rec [Ami]Tož dámo, co si dáte?
[E]Nic nechcu, pane vrchní, su smutná.
Tož něco si dát musíte, jste v jedničce.
[Ami]Říkám, nic nechcu, su smutná!
Tož si dajte něco tvrdýho, [Dmi]třeba gruziňak.
[Ami]Nechcu, pane vrchní, [E7]po tem su [Ami]smutná.

\ref
A Morava…
