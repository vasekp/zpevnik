\pisen{Hop Trop}{Překvápko}

\sloka
[C]Zaslech’ [Ami]jsem, že [Dmi]tuto[G7]vě v [C]Praze [Ami]pět na [Dmi]Smícho[G7]vě
[C]někdo [Ami]střelí [Dmi]indi[G7]ánskou [C]loď, [C7]
hned [F]napadlo mě [C7]bleskově, že [F]po důkladný [C7]opravě
s ní [F]překvapím svou [C7]milovanou [F]choť. [G]

\sloka
Já nezaváhal chviličku, co chtěl by za tu věcičku,
hned s majitelem vyjednávám sumu.
Říkal, že díky požáru má jen tu loď a kytaru
a k ní mu chybí aspoň doušek rumu.

\sloka
Tak do nejbližší Jednoty já začal nosit hodnoty,
byly toho dvě narvaný tašky,
už mám tě, moje kocábko, pro moji drahou překvápko
mě stálo tenkrát čtyřicet dvě flašky.

\ref
[G]Poplujem [D7]spolu tam dolů tou peřejí,
přestože [G]vodákům v ČSD nepřejí,
řeka nám [C]píseň [Cmi]bude [G]hrát,
že lodě [D7]nechtěj’ nikde [G]brát.

\sloka
Den nato jsem pak z moskviče, co půjčujou nám rodiče,
vyndal dva tři nepotřebný díly,
co neudělat pro lásku, já pro tebe, můj vobrázku,
se nerozmejšlím nikdy ani chvíli.

\sloka
Šoupnul jsem do svý aktovky dvě fungl nový mlhovky,
pár součástek a taky litřík Arvy,
vše vyměnil za lepidlo, pět latí, šrouby, tužidlo
a hlavně pikslu bleděmodrý barvy.

\sloka
Už starodávná tradice říká, že právě Lužnice
vodáků je každoročně Mekkou,
v den D u mostu v Suchdole: "No to je voda, tý vole!",
my zírali nad rozvodněnou řekou.

\sloka
Aby ti, co koukaj’ okolo, nám neříkali "prďolo",
my do svý lodi flegmaticky vlezli,
že přišlo velký koupání a vo šutráky drncání,
to nevadí, aspoň jsme se svezli!

\ref
Koukáme spolu tam dolů tou peřejí
na trosku lodě, co volně proud nese ji,
řeka nám píseň bude hrát,
zbyla nám žížeň akorát.
