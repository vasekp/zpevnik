\pisen{J. Nohavica}{Cukrářská bossa-nova}

\sloka
Můj [C\maj7]přítel [C#\dim]snídá sedm [Dmi7]kremrolí [G7]
a když je [C\maj7]spořádá, dá si [C#\dim]repete, [Dmi7]cukrlát[G7]ko,
on totiž [C\maj7]říká, "[C#\dim]dobré lidi zuby [Dmi7]nebolí, [G7]
a je to [C\maj7]paráda, chodit [C#\dim]po světě a [Dmi7]mít, [G7]
mít v ústech [C\maj7]sladko." [C#\dim\ Dmi7\ G7]

\ref
[Akordy stále dokola]Sláva, cukr a káva a půl litru Becherovky,
hurá, hurá, hurá, půjč mi bůra,
útrata dnes dělá čtyři stovky.
Všechny cukrářky z celé republiky
na něho dělají slaďounké cukrbliky
a on jim [Emi7]za odměnu zpívá [A7]zas a znovu
[Dmi7]tuhletu [G7]cukrářskou bossa-novu. [C\maj7\ C#\dim\ Dmi7\ G7]

\sloka
Můj přítel Karel pije šťávu z bezinek,
říká, že nad ni není, že je famózní, glukózní,
monstrózní, ať si taky dám.
Koukej, jak mu roste oblost budoucích maminek
a já mám podezření, že se zakulatí jako míč
a až ho někdo kopne, odkutálí se mi pryč
a já zůstanu sám, úplně sám.

\ref
Sláva…

\sloka
Můj přítel Karel Plíhal už na špičky si nevidí,
postava fortelná se mu zvětšuje, výměra tři ary,
on ale říká: glycidy jsou pro lidi,
je prý v něm kotelna, ta cukry spaluje,
někdo se zkáruje, někdo se zfetuje
a on jí bonpari, bon, bon, bon, bonpari.

\ref
Sláva…
