\pisen{K. Plíhal}{Jaro}

\sloka
[C]Jaro začlo [F]nešťast[C]ně, [Ami]jak os[F\maj7]tatně [G]vždyc[C]ky,
neslaně a [F]nemast[C]ně, zkrátka [Ami]makro[F\maj7\!GC]bioticky,
[F]slunce svítí [C]na dámy [G]čtoucí [Ami7]český [G]Burdy,
[C]stejně jak na [F]Saddá[C]my [Ami]stříle[F\maj7]jící\ \ \ [G]Kur[C]dy.

\sloka
Přišel jsem dřív z oběda, v pokoji hrál Verdi,
načapal jsem souseda, měl gatě na půl žerdi,
jdu do polí za prací a za druhou mízou,
unavený lustrací a parlamentnou krízou.

\sloka
Svítí křídla bělásků, v řece se třou líni,
na tradiční pomlázku s řetězy jdou skini.
Čichám děsný aroma, potácím se, blednu,
narazil jsem na Roma, co píchli ho tu v lednu.

\hvezda
Jaro začlo nešťastně, jak ostatně vždycky,
neslaně a nemastně, makrobioticky.
