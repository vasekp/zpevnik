\pisen{K. Kryl}{Morituri te salutant}

\sloka
Cesta je [Ami]prach a [G]šterk a [Dmi]udusaná [Ami]hlína
[C]a šedé [F]šmouhy [G7]kreslí do vla[C]sů
|: a z hvězdných [Dmi]drah má [G]šperk, co [C]kamením se [E]spíná,
[Ami]a pírka [G]touhy z [Emi]křídel pega[Ami]sů. :|

\sloka
Cesta je bič, je zlá jak pouliční dáma,
má v ruce štítky a v pase staniol
|: a z očí chtíč jí plá, když háže do neznáma
dvě křehké snítky rudých gladiol. :|

\ref
Seržante, [G]písek je bílý jak paže Daniely –
[Ami]počkejte chvíli, mé oči uviděly
[G]tu strašně dávnou vteřinu zapomnění –
[Ami]seržante, mávnou [G7]a budem zasvěceni!
[C]Morituri te salutant, [E]morituri te salutant…

\sloka
Tou cestou dál jsem šel, kde na zemi se zmítá
a písek víří křídla holubí
|: a marš mi hrál zvuk děl, co uklidnění skýtá,
a zvedá chmýří, které zahubí. :|

\sloka
Cesta je tér a prach a udusaná hlína,
mosazná včelka od vlkodlaka,
|: rezavý kvér, můj brach, a sto let stará špína
a děsně velká bílá oblaka. :|

\ref
Seržante…
