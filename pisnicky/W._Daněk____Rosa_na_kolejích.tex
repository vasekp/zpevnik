\pisen{W. Daněk}{Rosa na kolejích}

\sloka
[D]Tak jako jazyk [G6]stále na[G#6 A6]ráží
na vylomený [D]zub,
tak se vracím k [G6]svýmu ná[G#6]dra[A6]ží,
abych šel zas [D]dál.
Přede mnou [G6]stíny se [A6]plouží
a [Hmi]nad krajinou [D\dim]krouží
podivnej [G6 G#6]pták, \quad [A6]pták nebo [D]mrak.

\ref
Tak do toho [G6]šlápni, ať [A6]vidíš kousek [D]světa,
vzít do dlaní [G6]dálku [A6]zase jednou [D]zkus,
telegrafní [G6]dráty [A6]hrajou ti už [D]léta
to nekonečně [G6]dlou[G#6]hý [A6]mono[G#6]tón[G6]ní [D]blues.
Je []ráno, je ráno,
|:nohama [G6]stí[G#6]ráš [A6]rosu na [G#6 G6 D]kolejích.:|

\sloka
Pajda dobře hlídá pocestný, co se nocí toulaj,
co si radši počkaj, až se stmí, a pak šlapou dál.
Po kolejích táhnou bosí a na špagátu nosí
celej svůj dům – deku a rum.

\ref
Tak do toho šlápni…
