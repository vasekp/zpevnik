\pisen{J. Nohavica}{Dokud se zpívá, ještě se neumřelo}

\sloka
Z [C]Těšína [Emi]vyjíždí [Dmi7]vlaky co [F]čtvrthodi[C]nu, [Emi Dmi7 G]
[C]včera jsem [Emi]nespal a [Dmi7]ani dnes [F]nespoči[C]nu, [Emi Dmi7 G]
[F]svatý Me[G]dard, můj pa[C]tron, ťuká si na če[G]lo,
ale [F]dokud se [G]zpívá, [F]ještě se [G]neumře[C]lo. [Emi Dmi7 G]

\sloka
Ve stánku koupím si housku a slané tyčky,
srdce mám pro lásku a hlavu pro písničky,
ze školy dobře vím, co by se dělat mělo,
ale dokud se zpívá, ještě se neumřelo.

\sloka
Do alba jízdenek lepím si další jednu,
vyjel jsem před chvílí, konec je v nedohlednu,
za oknem míhá se život jak leporelo,
ale dokud se zpívá, ještě se neumřelo.

\sloka
Stokrát jsem prohloupil a stokrát platil draze,
houpe to, houpe to na housenkové dráze,
i kdyby supi se slítali na mé tělo,
tak dokud se zpívá, ještě se neumřelo.

\sloka
Z Těšína vyjíždí vlaky až na kraj světa,
zvedl jsem telefon a ptám se: "Lidi, jste tam?"
A z veliké dálky do uší mi zaznělo,
|:že dokud se zpívá, ještě se neumřelo.:|
