\pisen{Radůza}{Studený nohy}

\napis{P.}
|: [*A D E H A D E] :|

\sloka
[F#mi\ C#mi]Prší,\ \ \ \ \ \ \ \ \ [Hmi]choulím se [C#mi]do svrchníku,
[F#mi]než se [C#mi]otočím [Hmi]na pod[C#mi]patku
[F#mi]zalesk[C#mi]nou\ \ [Hmi]se světla [C#mi]na chodníku,
[F#mi]jak pět[C#mi]ka na věčnou [Hmi]oplát[C#mi]ku.

\sloka
Slyším kroky zakletejch panen,
to je vínem, to je ten pozdní sběr,
každá kosa najde svůj kámen,
to je vínem, ber mě, ber.

\ref
Studený [D]nohy [C#]schovám doma [F#mi]pod peřinou
a ráno [D]kafe dám si [C#]hustý jako [F#mi]tér,
přežiju [D]tuhle nedě[C#]li tak jako [F#mi]každou jinou,
na koho [D]slovo padne, [C#]ten je soli[H]tér. [F#mi]

\sloka
Broukám si píseň o klokočí,
prší a dlažba leskne se,
je chladno a hlava, ta se točí,
jak světla na plese.

\ref
Studený nohy…

\sloka
Tak mám a nebo nemám kliku,
zakletá panna směje se
a moje oči, lesknou se na chodníku,
jak světla na plese.

\ref
Studený nohy… 2\x
