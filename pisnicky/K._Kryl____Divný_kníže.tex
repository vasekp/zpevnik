\pisen{K. Kryl}{Divný kníže}

\sloka
[Ami]Jel krajem divný [C]kníže [Ami G]a chrpy povad[Ami]ly,
když z prstů koval [C]mříže [Ami]a z paží zábrad[E]lí,
|: on z [Ami]vlasů pletl [G]dráty, [Ami]měl kasematy z [G]dlaní [F\ Emi]
[Ami]a hadry za bro[G]káty, [Ami\ Emi]zlá slova místo [Ami]zbraní. :|

\sloka
Kam šlápl, vyrůstaly jen ocúny a blín,
když slzy nezůstaly, tak pomohl jim plyn,
|: hnal vítr plevy z polí a Kristus křičel z kříže,
když rány léčil solí ten prapodivný kníže. :|

\sloka
On pánem byl i sluhou, a svazek ortelů
si svázal černou stuhou, již smáčel v chanelu,
|: a hluchá píseň slábla, když havran značil cestu,
již pro potěchu ďábla vyhlásil v manifestu. :|

\sloka
Měl místo básní spisy a jako prózu mor
a potkany a krysy a difosgen a chlor,
|: měl klobouk z peří rajky a s důstojností snoba
on vymýšlel si bajky, v nichž vítězila zloba. :|

\sloka
Měl pendrek místo práva a statky pro gardu,
v níž vrazi řvali sláva pro rudou kokardu,
|: on lidem spílal zrádců, psal hesla do podloubí,
v nichž podle vkusu vládců lež s neřestí se snoubí. :|

\sloka
Dál kníže nosí věnce tou zemí zděšenou,
on strach má za spojence, jde s hlavou svěšenou
|: a netuší, že děti z té země, v které mrazí,
prostě a bez závěti mu jednou hlavu srazí. :|
