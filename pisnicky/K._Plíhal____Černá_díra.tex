\pisen{K. Plíhal}{Černá díra}

\sloka
[G]Mívali jsme [D]dědečka, [C]starého už [G]pána,
stalo se to v [D]červenci [C]jednou časně [D]zrá[G]na.
[Emi]Šel do sklepa [C]pro vidle, [A]aby seno [D]sklízel,
[G]už se ale [D]nevrátil, [C]prostě někam [D]zmi[G]zel.

\sloka
Máme doma ve sklepě malou černou díru,
na co přijde, sežere, v ničem nezná míru.
Nechoď, babi, pro uhlí, sežere i tebe,
už tě nikdy nenajdou příslušníci VB.

\sloka
Přišli vědci zdaleka, přišli vědci zblízka,
babička je nervózní a nás, děti, tříská.
Sama musí poklízet, běhat kolem plotny,
a děda je ve sklepě nekonečně hmotný.

\sloka
Hele, babi, nezoufej, moje žena vaří
a jídlo se jí většinou nikdy nepodaří.
Půjdu díru nakrmit zbytky od oběda,
díra všechno vyvrhne, i našeho děda.

\sloka
Tak jsem díru nakrmil zbytky od oběda,
díra všechno vyvrhla, i našeho děda.
Potom jsem ji rozkrájel motorovou pilou,
opět člověk zvítězil nad neznámou silou.

\hvezda
[A]Dědeček se [E]raduje, [D]že je zase v [A]penzi,
teď je naše [E]písnička [D]zralá pro re[E]cen[A]zi.
