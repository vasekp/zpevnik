\pisen{J. Nohavica}{Těšínská}

\sloka
[Ami]Kdybych se narodil [Dmi]před sto lety [F E]v tomhle [Ami]městě, [Dmi F E Ami]
u Larischů na zahradě [Dmi]trhal bych květy [F E]své ne[Ami]věstě. [Dmi F E Ami]
[C]Moje nevěsta by [Dmi]byla dcera ševcova
z [F]domu Kamiňskich [C]odněkud ze Lvova,
kochal bym ja i [Dmi]pieščil, [F E]chyba lat [Ami]dwieščie. [Dmi F E Ami]

\sloka
Bydleli bychom na Sachsenbergu v domě u žida Kohna,
nejhezčí ze všech těšínských šperků byla by ona.
Mluvila by polsky a trochu česky,
pár slov německy a smála by se hezky.
Jednou za sto let zázrak se koná, zázrak se koná.

\sloka
Kdybych se narodil před sto lety, byl bych vazačem knih,
u Prohazků dělal bych od pěti do pěti a sedm zlatek za to bral bych,
měl bych krásnou ženu a tři děti,
zdraví bych měl a bylo by mi kolem třiceti,
celý dlouhý život před sebou, celé krásné dvacáté století.

\sloka
Kdybych se narodil před sto lety v jinačí době,
u Larischů na zahradě trhal bych květy, má lásko, tobě.
Tramvaj by jezdila přes řeku nahoru,
slunce by zvedalo hraniční závoru
a z oken voněl by sváteční oběd.

\sloka
Večer by zněla od Mojzese melodie dávnověká,
bylo by léto tisíc devět set deset, za domem by tekla řeka,
vidím to jako dnes: šťastného sebe,
ženu a děti a těšínské nebe,
ještě že člověk nikdy neví, co ho čeká,
[Ami Dmi F E Ami Dmi F E Ami]na na na…
