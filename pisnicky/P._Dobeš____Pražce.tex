\pisen{P. Dobeš}{Pražce}

\sloka
[C]Házím tornu na svý záda, feldflašku a [G7]sumky,
navštívím dnes kamaráda z železniční průmky.

\ref
Vždyť je [C]jaro, zapni si kšandy,
pozdravuj [G7]vlaštovky, a muziko, ty [C]hraj.

\sloka
Vystupuji z vlaku, který mizí v dálce,
stojím v České Třebové a všude kolem pražce.

\ref
…

\sloka
Pohostil mě slivovicí, představil mě Mařce,
posadil mě na lavici z dubového pražce.

\sloka
Provedl mě domem – nikde kousek zdiva,
všude samej pražec, jen Máňa byla živá.

\ref
To je to jaro, zapni si kšandy,
pozdravuj vlaštovky, a muziko, ty hraj.

\sloka
Plakáty nás informují: "Přijď pracovat k dráze,
pakliže ti vyhovují rychlost, šmír a saze."

\sloka
A jestliže jsi labužník a přes kapsu se praštíš,
upečeš i krávu na železničních pražcích.

\sloka
A naučíš se skákat tak, jak to umí vrabec,
když na nohu si pustíš železniční pražec.

\sloka
Když má děvče z Třebové rádo svého chlapce,
posílá mu na vojnu železniční pražce.

\sloka
A když děti zlobí, tak hned je doma mazec,
Děda Mráz jim nepřinese ani jeden pražec.

\sloka
Před děvčaty z Třebové chlubil jsem se silou,
pozvedl jsem pražec, načež odvezli mě s kýlou.

\sloka
Pamatuji pouze ještě operační sál,
pak praštili mě pražcem a já jsem tvrdě spal.

\ref
A bylo jaro, zapni si kšandy,
lítaly vlaštovky a zelenal se háj.
