\pisen{J. Nohavica}{Margita}

\sloka
[Emi]Točí se, točí [Ami]kolo [Emi]dokola a až nás [Hmi]smrtka [C]zavolá,
tak [G]vytáhneme [Emi]rance,
sbalíme svý tě[Ami]lesný [Emi]ostatky, srazíme [Hmi]podpat[C]ky
a po[G]tom dáme se do [Emi]tance.
[Ami]Tančí se tradič[Hmi]ně\ \ [Emi]gavotta na konci [Hmi]živo[C]ta,
na konci [G]tohohle [Emi]plesu.
Promiňte, že vám [Ami]šlapu [Emi]na nohu, já za to [Hmi]nemo[C]hu,
já ta[G]dy vlastně ani [Emi]nejsu.

\ref
[Ami]A krásná Margita, [Emi]tanečnice smrti,
[Ami]má svetr vzoru pepi[Emi]ta a zadkem vrtí,
[Ami]začíná maškarní [Emi]ples,
kams’ to [D]vlez’, kams’ to [Emi]vlez’,
kams’ to [H7]vlez’, kams’ to [Emi]vlez’?

\sloka
Nějakej dobrák stáhnul roletu, na tomhle parketu
je vidět jenom na půl metru,
tváře se ztrácejí v šerosvitu, každý hledí na Margitu
v pepitovým svetru,
kapelník pozvednul taktovku, vytáhnul aktovku
a v ní měl nějaký noty,
tváře se točí v tombole, někdo si sedí za stolem,
jiný se na parketu potí.

\ref
A krásná Margita…

\sloka
Pokojská v šest přijde do práce, provětrá matrace,
pod lůžkem najde zlatou minci.
Děvenka dole v recepci čte si o antikoncepci
a přemýšlí o krásném princi.
Šatnářce v šatně zbyly dvě vesty, no to je neštěstí,
a co když přijde ráno kontrola?
Ručičky ukazují čtyři nula nula, jedna z nich se hnula,
vše se točí dokola.

\ref
A krásná Margita…
končí se maškarní ples,
už je dnes, už je dnes,
už je dnes, už je dnes.
