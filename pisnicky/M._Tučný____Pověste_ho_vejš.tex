\pisen{M. Tučný}{Pověste ho vejš}

\rec
Na dnešek jsem měl divnej sen: slunce pálilo a před saloonem stál v prachu dav, v tvářích cejch očekávání. Uprostřed  popraviště z~hrubých klád  šerifův pomocník sejmul odsouzenci z hlavy kápi a dav zašuměl překvapením. I já jsem zašuměl překvapením: ten odsouzenec jsem byl já a šerif četl neúprosným hlasem rozsudek: 

\ref
Pověste ho [Emi]vejš, ať se houpá, pověste ho [G]vejš, ať má [D]dost,
pověste ho [Ami]vejš, ať se [Emi]houpá, že tu [D]byl nezvanej [Emi]host.

\sloka
Pověste ho, že byl jinej, že tu s náma dejchal stejnej vzduch,
pověste ho, že byl línej a tak trochu dobrodruh.

\hvezda
Pověs[Emi]te ho za El Paso, za Snídani v [G]trávě a Lodní [D]zvon,
za to, že [Ami]neoplýval [Emi]krásou
a že měl [C]country rád a že se [H7]uměl smát i [Emi]vám.

\jine{R2.}
Nad hla[G]vou mi slunce [D]pálí, konec [Ami]můj nic neod[G]dá[D]lí,
do mých [G]snů se dívám [D]zdáli
a [Ami]do uší mi stále zní tahle [H7]píseň poslední.

\sloka
Pověste ho za tu banku, v který zruinoval svůj vklad,
za to, že nikdy nevydržel na jednom místě stát.

\jine{R2.}
Nad hlavou mi slunce pálí, konec můj nic neoddálí …

\ref
Pověste ho vejš…

\sloka
Pověste ho za tu jistou, který nesplnil svůj slib,
že byl zarputilým optimistou, a tak dělal spoustu chyb.

\hvezda
Pověste ho, že se koukal a že hodně jedl a hodně pil,
že dal přednost jarním loukám,
a pak se oženil a pak se usadil a žil.

\ref
Pověste ho vejš… {\sl (do ztracena)}
