\pisen{K. Plíhal}{Kluziště}

\sloka
[C]Strejček [Emi7\!/H]kovář\ \ \ [Ami7]chytil [C/G]kleště,
[Fmaj7]uštíp’ z [C]noční [Fmaj7 G]oblohy
[C]jednu [Emi7\!/H]malou\ \ \ [Ami7]kapku [C/G]deště,
[Fmaj7]ta mu [C]spadla [Fmaj7]pod no[G]hy,
[C]nejdřív [Emi7\!/H]ale\ \ \ \ \ \ \ [Ami7]chytil [C/G]slinu,
[Fmaj7]tak šáh’ [C]kamsi [Fmaj7]pro pi[G]vo,
[C]pak při[Emi7\!/H]táhl\ \ \ \ \ \ [Ami7]kovad[C/G]linu
[Fmaj7 C]a obrovský [Fmaj7 G]kladivo.

\ref
Zatím [C]tři bílé [Emi7\!/H]vrány\ \ \ [Ami7]pěkně za se[C/G]bou
kolem [Fmaj7]jdou, někam [C]jdou, do ryt[D7]mu se kýva[G]jí,
tyhle [C]tři bílé [Emi7\!/H]vrány\ \ \ [Ami7]pěkně za se[C/G]bou
kolem [Fmaj7]jdou, někam [C]jdou, nedo[Fmaj7]jdou, nedo[C]jdou.

\sloka
Vydal z hrdla mocný pokřik ztichlým letním večerem,
pak tu kapku všude rozstřík’ jedním mocným úderem,
celej svět byl náhle v kapce a vysoko nad námi
na obrovské mucholapce visí nebe s hvězdami.

\sloka
Zpod víček mi vytrysk’ pramen na zmačkané polštáře,
kdosi mě vzal kolem ramen a políbil na tváře,
kdesi v dálce rozmazaně strejda kovář odchází,
do kalhot si čistí ruce umazané od sazí.
