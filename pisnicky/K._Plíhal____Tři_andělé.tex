\pisen{K. Plíhal}{Tři andělé}

\sloka
V [Dmi]zahradě pod vrbou smuteč[Ami]ní
[Dmi]sedě[C]li [B]tiše jak [A]pěna
tři [Dmi]andělé – dva byli skuteč[Ami]ní,
[Dmi]ten tře[C]tí [B]byla má [Ami\ Dmi]žena.
|: Při[G]sednout ne[C]měl jsem od[Dmi]vahu,
ml[G]čeli pa[C]trně o [Dmi]mně,
duši mám čistou jak podla[Ami]hu
[Dmi]po ple[C]se v [B]Národním [Ami\ Dmi]domě. :|

\sloka
Určitě skončili u cifer, sčítajíc všechny mé hříchy,
až si to přebere Lucifer, nejspíš se potrhá smíchy.
|: Obloha zčernala sazemi z komínů vesmírných lodí,
dokud jsou andělé na Zemi, nic zlého se nepřihodí… :|
