\pisen{K. Kryl}{Veličenstvo Kat}

\sloka
V [Ami]ponurém osvětlení [G]gotického [Ami]sálu
[C]kupčíci vyděšení [Dmi]hledí do mi[E]sálů
[Ami]a houfec [F]mordýřů si [G]žádá požeh[C]ná[G]ní,
|: [Dmi]vždyť první z [Ami]rytířů je [Emi]Veličenstvo [Ami]Kat. :|

\sloka
Kněz–ďábel, co mši slouží, z oprátky má štolu,
pod fialovou komží láhev vitriolu,
pach síry z hmoždířů se valí k rudé kápi
|: prvního z rytířů, hle: Veličenstvo Kat. :|

\ref
[C]Na korouhvi [G]státu [F]je emblém s [G]gilotinou,
z [C]ostnatýho [G]drátu [F]páchne to [G]shnilotinou,
v [Dmi]kraji hnízdí hejno krkav[Ami]čí,
[Dmi]lidu vládne mistr poprav[E7]čí.

\sloka
Král klečí před Satanem, na žezlo se těší
a lůza pod platanem radu moudrých věší
a zástup kacířů se raduje a jásá,
|: vždyť prvním z rytířů je Veličenstvo Kat. :|

\sloka
Na rohu ulice vrah o morálce káže,
před vraty věznice se procházejí stráže,
z vojenských pancířů vstříc černý nápis hlásá,
|: že prvním z rytířů je Veličenstvo Kat. :|

\zlom

\ref
Nad palácem vlády ční prapor s gilotinou,
děti mají rády kornouty se zmrzlinou,
soudcové se na ně zlobili,
zmrzlináře dětem zabili.

\sloka
Byl hrozný tento stát, když musel jsi se dívat,
jak zakázali psát a zakázali zpívat,
a bylo jim to málo, poručili dětem
|: modlit se, jak si přálo Veličenstvo Kat. :|

\sloka
S úšklebkem Ďábel viděl pro každého podíl,
syn otce nenáviděl, bratr bratru škodil,
jen motýl smrtihlav se nad tou zemí vznáší,
|: kde v kruhu tupých hlav dlí Veličenstvo Kat. :|
