\pisen{K. Kryl}{Píseň Neznámého vojína}

\rec
Zpráva z tisku: "Obě delegace položily pak věnce na hrob Neznámého vojína."\ A co na to Neznámý vojín?

\sloka
V [Ami]čele klaka, pak ctnostné rodiny a náruč chryzan[E7]tém,
černá saka a žena hrdiny pod paží s aman[Ami]tem, [E7]
[Ami]kytky v dlaních a pásky smuteční civí tu před bra[E7]nou,
ulpěl na nich pach síně taneční s bolestí sehra[Ami]nou.

Co tady [F]čumíte? Vlezte mi [G]někam!
Copak si [Ami]myslíte, že na to [G]čekám?
Co tady [Ami]civíte? Táhněte [G7]domů!
[Ami]Pomníky stavíte, [F]prosím vás, [E7]komu? [Ami E]

\sloka
Jednou za čas se páni ustrnou a přijdou poklečet,
je to trapas, když s pózou mistrnou zkoušejí zabrečet,
pak se zvednou a hraje muzika písničku mizernou,
ještě jednou se trapně polyká nad hrobem s lucernou.

Co tady civíte? Zkoušíte vzdechnout,
copak si myslíte, že jsem chtěl zdechnout?
Z lampasů je nám zle, proč nám sem leze? 
Kašlu vám na fangle! Já jsem chtěl kněze!

\sloka
Nejlíp je mi, když kočky na hrobě v noci se mrouskají,
ježto s těmi, co střílej’ po sobě, vůbec nic nemají,
mňoukaj’ tence a nikdy neprosí, neslouží hrdinům,
žádné věnce pak na hrob nenosí Neznámým vojínům.

Kolik vám platějí za tenhle nápad?
Táhněte raději s děvkama chrápat!
Co mi to říkáte? Že šel bych zas? Rád?
[Ami]Odpověď čekáte? [F]Nasrat, jo, [E]nasrat!
