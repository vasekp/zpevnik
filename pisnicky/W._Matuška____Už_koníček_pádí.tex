\pisen{W. Matuška}{Už koníček pádí}

\sloka
Znám [G7]zem plnou [C]mlíka a buclatejch [G]krav,
kde proud řeky [D7]stříká na dřevěnej [G]splav.
Mám [G7]jediný [C]přání, snům ostruhy [G]dát
a pod známou [D7]strání zas kuličky [G]hrát.

\ref
Už koníček pádí a zůstane stát
až v tý zemi mládí, kde já žiju rád.
A slunce tam pálí a pořád je máj
a ceny jsou stálý a lidi se maj’.

\sloka
Kde všechno je známý, zvuk tátovejch bot
a buchty mý mámy a natřenej plot.
Z něj barva už prejská, ale mně je to fuk,
já, když se mi stejská, jsem jak malej kluk.

\ref
Už koníček…

\sloka
V tý zemi jsou lípy a ve květech med
a u sudů pípy a u piva led.
A okurky v láku a cestovní ruch
a hospod jak máku a holek jak much.

\ref
A slunce tam pálí a pořád je máj
a ceny jsou stálý a lidi se maj’.
Už koníček pádí a zůstane stát,
až v tý zemi mládí, kde žiju tak rád.
