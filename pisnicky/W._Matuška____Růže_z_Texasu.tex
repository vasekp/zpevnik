\pisen{W. Matuška}{Růže z Texasu}

\sloka
Je[D]du vám takhle [D7]stezkou dát [G]stádu k řece [D]pít,
v tom potkám holku [Hmi]hezkou, až [E7]jsem vám z koně [A7]slít.
Měla [D]kytku žlutejch [D7]květů, snad [G]růží, co já [D]vím,
znám plno hezkejch [H7]ženskejch k světu, ale [Emi]tahle [A7]hraje [D]prim.

\ref
[D7]Kdo si [G]kazíš smysl pro krásu, ať s [D]tou a nebo s tou,
dej si říct, že kromě [Hmi]Texasu tyhle [E7]růže neros[A7]tou.
Ať máš [D]kolťák nízko [D7]u pasu, ať jsi [G]třeba zloděj [D]stád,
svoji žlutou růži z [H7]Texasu budeš [Emi]pořád [A7]mít už [D]rád.

\sloka
Řekla, že tu žije v ranči, jen sama s tátou svým
a že hrozně ráda tančí, teď zrovna nemá s kým,
tak já jsem se jí nabíd, že půjdu s ní a rád
a že se dám i zabít, když si to bude přát.

\ref
Kdo si…

\sloka
Hned si dala se mnou rande a přišla přesně v půl
a dole teklo Rio Grande a po něm měsíc plul.
Když si to tak v hlavě srovnám, co víc jsem si moh’ přát,
ona byla krásná, štíhlá, rovná, zkrátka akorát.

\ref
Kdo si…

\sloka
Od těch dob svý stádo koní sem vodím k řece pít
a žiju jenom pro ni a chtěl bych si ji vzít.
Když večer banjo ladím a zpívám si tu svou,
tak v duchu pořád hladím tu růži voňavou.

\ref
Kdo si…
