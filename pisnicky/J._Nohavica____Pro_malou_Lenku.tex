\pisen{J. Nohavica}{Pro malou Lenku}

\sloka
[Ami]Jak mi tak docházejí [D7]síly, já [G]pod jazykem [G/F#]cítím [Emi]síru,
[Ami]víru, ztrácím [D7]víru, a to mě [G]míchá,
[Ami]minomety reflektorů [D7]střílí, jsem malým
[G]terčem [G/F#]na bitevním [Emi]poli,
kdekdo mě [Ami]skolí, a to mě [D7]bolí, u srdce [G]píchá.

\ref
[Ami]Rána jsou smutnější než [D7]večer,
z rozbitého [G]nosu [G/F#]krev mi [Emi]teče,
na čísle [Ami]5610[D7]9 nikdo to [G]nebere,
[Ami]rána jsou smutnější než [D7]večer,
na hrachu [G]klečet, [G/F#]klečet, [Emi]klečet,
[Ami]opustil mě můj děd [D7]Vševěd a zas je [G]úterek.
[Ami]Jak se ti [D7]vede? No [G]někdy fajn a [G/F#]někdy je to v [Emi]háji,
dva [Ami]pozounisti z vesnické [D7]kutálky pod okny mi [G]hrají:
[Ami D7 G Emi Ami D7 G Ami D7 G Emi Ami D7 G]tú tú…

\zlom

\sloka
Jak říká kamarád Pepa:
"co po mně chcete, slečno z první řady,
vaše vnady mě nebaví a trošku baví,"
sudička moje byla slepá, když mi řekla to, co mi řekla,
píšou mi z pekla, že prý mě zdraví, že prý mě zdraví.

\ref
Rána…

\sloka
Má malá Lenko, co teď děláš,
chápej, že čtyři roky, to jsou čtyři roky
a čas pádí a já jsem tady a ty zase jinde,
až umřem, říkej, žes’ nás měla,
to pro tebe skládáme tyhle sloky,
na hrachu klečíme a hloupě brečíme a světu dáváme kvinde.

\ref
Rána…
