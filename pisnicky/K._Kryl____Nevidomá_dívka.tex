\pisen{K. Kryl}{Nevidomá dívka}

\sloka
V [C]zahradě [Dmi]za cihlovou [C]zídkou, [Dmi]
[C]popsanou v [Dmi]slavných výro[C]čích, [Dmi]
[C]sedává [Dmi]na podzim [E]na trávě [Ami]před besídkou
[F]děvčátko s [G]páskou na o[E]čích.

\sloka
Pohádku o mluvícím ptáku
nechá si přečíst z notesu,
pak pošle polibek po chmýří na bodláku
na vymyšlenou adresu.

\ref
Prosím vás, [Ami]nechte ji, ach [Dmi]nechte ji,
[Ami]tu nevidomou [G]dívku,
[Dmi]prosím vás, [G]nechte ji si [E]hrát,
[Dmi]vždyť možná hraje si [Ami]na slunce s nebesy,
[F]jež nikdy neuvidí, [G]ač ji bude [E]hřát.

\sloka
Pohádku o mluvícím ptáku
a o třech zlatých jabloních
a taky o lásce, již v černých květech máku
přivezou jezdci na koních.

\sloka
Pohádku o kouzelném slůvku,
jež vzbudí všechny zakleté,
pohádku o duze, jež spává na ostrůvku,
na kterém poklad najdete.

\ref
Prosím vás, nechte ji…

\rec
Rukama dotýká se květů,
a neruší ji motýli,
jen trochu hraje si s řetízkem amuletu,
jen na chvíli, jen na chvíli.

\ref
Prosím vás, nechte ji…
