
\pisen{J. Nohavica}{Osud}

\sloka
[Hmi]Papoušek arara [D]zobáčkem z krabičky
[A]vytáh mi na pouti los, stálo tam:
"[Hmi]Čeká tě kytara, [G]sláva a písničky,
[F#7]smůla a rozbitý nos". Řek’ jsem mu:
"[Hmi]Máš krásné peří, [D]papoušku, hlupáčku,
[A]pouťový čaroději, ale
[Hmi]kdo by dnes věřil [G]osudům za kačku,
[F#7]poukázkám na naději?"

\ref
[Hmi]Osud se nepíše přes kopí[A]rák,
[D]osud se [F#]na pouti [G]nevybí[F#]rá,
[Hmi]osud je malůvka [A]na lemu talíře,
[D]vystydlá polévka [F#7]chudého malíře,
[Hmi]malíře… [D A Hmi G F# Hmi]

\sloka
V zahradě za domem trnul jsem v úžasu,
když potom jednoho dne
spatřil jsem Salome s lilií u pasu,
jak ke mně pomalu jde. Srdce mi
spálila kopřiva, že něco nesmí se,
i když by se mělo dít,
pohřbený zaživa a s hlavou na míse,
bylo mi od toho dne těžko žít.

\zlom

\ref
Osud… je malůvka, vítr ji pomačká,
pečená brambora zmoklého pasáčka, pasáčka…

\sloka
S rukama nad hlavou, sám sobě rukojmí,
bojácný o vlastní strach
procházel Ostravou trhovec odbojný,
já mu šlapal po patách. Prodával
zelí a kedlubny, nohy nás bolely,
zvolna se rozpadal šat.
On bouchal na bubny, já tloukl činely,
tak jsem se naučil hrát.

\ref
Osud… je malůvka visící v kostele,
knoflíky po kapsách dobrého přítele, přítele…

\sloka
"Salome, vrať se mi," volal jsem do noci
květnaté přívaly vět,
"jsem nemocen písněmi, není mi pomoci,
horečka bortí můj svět, před bytem
stojí mi nožíři s cákanci na botách,
na zádech plátěný vak,
hraju jen na čtyři akordy života,
stíhám to jen tak, jen tak".

\ref
Osud… je malůvka hebká jak oblaka,
šaškovská čepice smutného zpěváka, zpěváka…

\sloka
Končí se dostihy smutku a radosti,
sázky se vyplácejí.
Napsal jsem do knihy přání a stížností
o svojí beznaději.
Žokej se pousmál, ukázal na koně:
"plemenný arabský chov".
Koni jsem cukřík dal, pak k senu přivoněl,
a zazpíval beze slov:

\ref
Osud… je malůvka, tisíc let nepřečká,
balónek vnoučete, hůlčička stařečka, stařečka…
