\pisen{K. Plíhal}{Dvě spálený srdce (Nagasaki, Hirošima)}

\sloka
[C]Tramvají [G]dvojkou [F]jezdíval jsem [G]do Žide[C G F G]nic,
z [C]tak veliký [G]lásky [F]většinou [G]nezbyde [Ami]nic,
z [F]takový [C]lásky [F]jsou kruhy [C]pod oči[G]ma
a [C]dvě spálený [G]srdce – [F]Nagasaki, [G]Hirošima. [C G F G]

\sloka
Jsou jistý věci, co bych tesal do kamene,
tam, kde je láska, tam je všechno dovolené,
a tam, kde není, tam mě to nezajímá,
jó, dvě spálený srdce – Nagasaki, Hirošima.

\sloka
Já nejsem svatej, ani ty nejsi svatá,
ale jablka z ráje bejvala jedovatá,
jenže hezky jsi hřála, když mi někdy bylo zima,
jó, dvě spálený srdce – Nagasaki, Hirošima.

\sloka
= 1. + |: a dvě spálený srdce – Nagasaki, Hirošima. :|
