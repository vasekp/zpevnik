\pisen{K. Plíhal}{Milej pane Dänikene}

\sloka
[A]Milej [H7]pane [E7]Däni[A]kene, [D7]přijeď[E7]te k nám [A]na Hanou.
[A]Olo[H7]mouc je v [E7]tomto [A]směru [D7]ještě [E7]nepro[A]bádanou.
[D]Na co byste [G]pořád lítal [D]za ufóny [H]do Peru?
[A]Zajeď[H7]te k nám [E7]na Ha[A]nou a [D7]já to s [E7]váma [A]proberu.
        
\sloka
Ve vinárně budem trávit velmi plodné večery,
velmi si je pochvaloval pan Clarke i pan Bradbury.
Po záchodcích budem zkoumat tajuplné symboly,
které mohou mimo jiné znamenati cokoli.

\sloka
Nakreslím vám plánek lodě, co mně vezme na Pluto,
bude stačit jen dát páku do polohy "Zapnuto,"
o další se postarají navigační roboti,
kočka se nám za tu cestu pětsetšestkrát okotí.

\sloka
Milej pane Dänikene, přijeďte k nám na Hanou.
Počkám na vás zejtra večer ve špitále za bránou.
Vy to víte, jak to bolí, když vám nikdo nevěří.
Já už končím, sestřička nás právě volá k večeři.
