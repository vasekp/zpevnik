\pisen{M. Dietrich}{Sag mir, wo die Blumen sind}

\sloka
[C]Sag mir, wo die [Ami]Blumen sind, [F]wo sind sie ge[G7]blieben?
[C]Sag mir, wo die [Ami]Blumen sind, [F]was ist ge[G7]schehen?
[C]Sag mir, wo die [Ami]Blumen sind, [F]Mädchen pflückten [G7]sie geschwind.
[F]Wann wird man [C]je verstehen, [F]wann wird man [G7]je ver[C]stehen?

\sloka
Sag mir, wo die Mädchen sind, wo sind sie geblieben?
Sag mir, wo die Mädchen sind, was ist geschehen?
Sag mir, wo die Mädchen sind, Männer nahmen sie geschwind.
Wann wird man je verstehen, wann wird man je verstehen?

\sloka
Sag mir, wo die Männer sind, wo sind sie geblieben?
Sag mir, wo die Männer sind, was ist geschehen?
Sag mir, wo die Männer sind, zogen fort, der Krieg beginnt.
Wann wird man je verstehen, wann wird man je verstehen?

\sloka
Sag, wo die Soldaten sind, wo sind sie geblieben?
Sag, wo die Soldaten sind, was ist geschehen?
Sag, wo die Soldaten sind, über Gräber weht der Wind.
Wann wird man je verstehen, wann wird man je verstehen?

\sloka
Sag mir, wo die Gräber sind, wo sind sie geblieben?
Sag mir, wo die Gräber sind, was ist geschehen?
Sag mir, wo die Gräber sind, Blumen wehen im Sommerwind.
Wann wird man je verstehen, wann wird man je verstehen?

\sloka
= 1.
