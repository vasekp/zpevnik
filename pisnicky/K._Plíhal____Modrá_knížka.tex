\pisen{K. Plíhal}{Modrá knížka}

\sloka
Celý [G]svět je náhle [Gmaj7]beznadějně [Ami7]šedý, [D7]
bloudím [Ami]ulicemi [Ami7+]zoufalý a [Ami7]bledý, [D7]
všude [G]vidím mládež [H7]odvedenou, [E7]veselou a [A7]bodrou,
[C]a jenom mně [C#dim]dali knížku [Ami7]modrou. [D7]

\sloka
I ten Prouza, co se pomočívá denně,
spěchá oznámit tu šťastnou zprávu ženě,
Václav zase v Mikulovské opíjí se s tchánem,
už se vidí nejmíň kapitánem.

\sloka
Všude v parcích vidím šťastné kamarády,
jak se seznamují s vojenskými řády,
mezi nima také Novák, no ten se skleněným okem,
zkouší chodit pořadovým krokem.

\sloka
Podroušenej Viktor tančí s Hugem kankán,
Viktor bude spojař, Hugo zase tankán,
a ten Jarda, kterej nedokončil ani osmiletku,
hodil po mně láhví se zvoláním "zmetku!"

\sloka
Nadarmo si říkám: "Karle, nešil,"
jenomže já se na tu vojnu tolik těšil,
marně u odvodu doktorovi obálku jsem dával,
prej mě nevezmou, že mají velkej nával.

\sloka
Jak to vysvětlím své milované Blance,
ráda se chlubívala, že má chlapce brance,
máma bude brečet, táta rozdupe mi hračky,
už se, chudák, tolik těšil na odznáčky.

\sloka
= 1.
+ [D7]modrou, [G]modrou.
