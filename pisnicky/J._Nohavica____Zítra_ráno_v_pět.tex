\pisen{J. Nohavica}{Zítra ráno v pět}

\sloka
Až [Ami]zítra ráno v pět mě [C]ke zdi postaví,
ješ[Dmi]tě si napos[G7]led dám [C]vodku na zdra[Ami]ví,
z očí [Dmi]pásku strhnu [G7]si, to abych [C]viděl na ne[Ami]be
a [Dmi]pak vzpomenu [E7]si, [Ami]lásko, na tebe,
ná [Dmi G C Ami]ná na ná…
a [Dmi]pak vzpomenu [E7]si na te[Ami]be.

\sloka
Až zítra ráno v pět přijde ke mně kněz,
řeknu mu, že se splet’, že mně se nechce do nebes,
že žil jsem, jak jsem žil, a stejně i dožiju
a co jsem si nadrobil, to si i vypiju.

\sloka
Až zítra ráno v pět poručík řekne "pal!",
škoda bude těch let, kdy jsem tě nelíbal,
ještě slunci zamávám, a potom líto přijde mi,
že tě, lásko, nechávám, samotnou tady na zemi.

\sloka
Až zítra ráno v pět prádlo půjdeš prát
a seno obracet, já u zdi budu stát,
tak přilož na oheň a smutek v sobě skryj,
prosím, nezapomeň, nezapomeň a žij.
