\pisen{J. Nohavica}{Tři čuníci}

\sloka
V [C]řadě za sebou tři čuníci jdou,
ťápají se v blátě cestou neces[Ami]tou,
[Dmi]kufry nemají, cestu nezna[G7]jí,
[Dmi]vyšli prostě do světa a [G7]vesele si zpívají:

\ref
[C Ami Dmi G7 Dmi G7]Uí ui ui ui uí…

\sloka
Auta jezdí tam, náklaďáky sem,
tři čuníci jdou jdou rovnou za nosem,
žito chroupají, ušima bimbají,
vyšli prostě do světa a vesele si zpívají:

\ref
Uí ui…

\sloka
Levá, pravá teď, přední, zadní už,
tři čuníci jdou, jdou, jako jeden muž.
Lidé zírají, důvod neznají,
proč ti malí čuníci tak vesele si zpívají.

\sloka
Když kopýtka pálí, když jim dojde dech,
sednou ku studánce na vysoký břeh.
Ušima bimbají, kopýtka máchají,
chvilinku si odpočinou a pak dál se vydají.

\sloka
Když se spustí déšť, roztrhne se mrak,
k sobě přitisknou se čumák na čumák.
Blesky blýskají, kapky pleskají,
oni v dešti, nepohodě vesele si zpívají.

\sloka
Za tu spoustu let, co je světem svět,
přešli zeměkouli třikrát tam a zpět
v řade za sebou, hele, támhle jdou!
Pojďme s nima zazpívat si jejich píseň veselou.
