\pisen{M. Kubišová}{Lucerna}

\sloka
Je [Dmi]půlnoc nádherná, [C]spí i [Ami]lucerna,
[B]tys’ mě opustil [A]ospalou,
tu v [Dmi]hloubi zahrady [C]cítím [Ami]úklady,
s [B]píšťalou [A]někdo sem [Dmi]kráčí.
[F]Hrá náramně [C]krásně a na mě
[B]tíha podivná [A]doléhá,
[F]hrá náramně, [C]zná mě, [Ami]nezná mě,
[Dmi]něha a [A]hudba až k [Dmi]pláči.

\sloka
Pak náhle pomalu skládá píšťalu,
krok, a slušně se uklání,
jsem rázem ztracená, co to znamená,
odháním strach, a on praví:
Pan jméno mé, mám už renomé,
Pan se jmenuji a jsem bůh,
Pan, bůh všech stád, vás má, slečno, rád,
jen Pan je pro vás ten pravý.

\sloka
Ráno, raníčko, ach, má písničko,
Pan mi zmizel i s píšťalou,
od Pana, propána, o vše obrána,
ospalou najde mě máti.
Hrál a ve tmě krásně podved’ mě,
kam jsem to dala oči, kam,
Pan, pěkný bůh, já teď nazdařbůh
počítám "dal" a "má dáti". 
