\pisen{K. Plíhal}{Fotka}

\sloka
[C]Kdosi mi [Emi]do kapsy dal [Ami]fotku jedné [Emi]slečny,
[C7]taky je [F]možné, že to [A7]sako moje [Dmi]není.
[E7]V každém [Ami]případě jsem [G7]za tu fotku [C]vděčný,
[H7]dívka je [Emi]pohledná a [H7]půvabně se [G]kření.

\sloka
"Svému Michalovi na památku Věra"
hůlkovým písmem je na zadní straně psáno.
Marně si vzpomínám, s kým popíjel jsem včera,
marně si vzpomínám, s kým popíjel jsem ráno.

\hvezda
[C]Tajemná [G#7]Věro, [D9]to, že nejsem [G]Michal,
není [C]důvod, abych [A7]vzdychal a v [D9]noci nemoh’ [G]spát,
chudák [C]Michal časem [C7]žasne, až ti [F]stářím zvadnou [Fmi]líce,
u mě [C]budeš krasa[G#7]vice [G7]napo[C]řád.
