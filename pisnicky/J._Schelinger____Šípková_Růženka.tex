\pisen{J. Schelinger}{Šípková Růženka}

\sloka
Ještě [Gmi]spí a spí a spí zámek [F]šípkový,
[Gmi]žádný princ tam v lesích [Dmi]ptáky neloví.
Ještě [Gmi]spí a spí a spí dívka [F]zakletá,
[Gmi]u lůžka jí planá [Dmi]růže rozkvétá.

\ref
[B]To se [C]schválně dětem [Gmi]říká
[Eb]aby s [F]důvěrou šly [B]spát, [D7]klidně spát,
že se [Gmi]dům probou[F]zí
a ta [B]kráska proci[Eb]tá…
[Cmi]Zatím spí tam [Dmi]dál, spí tam v [Gmi]růžích.

\sloka
Kdo jí ústa k ústum dá, kdo ji zachrání,
kdo si dívku pobledlou vezme za paní?
Vyjdi zítra za ní a nevěř pohádkám,
žádný princ už není, musíš tam jít sám.

\ref
To se schválně…
[Gmi]Musíš přijít [F]sám,
nesmíš [B]věřit pohád[Eb]kám,
|:[Cmi]čeká dívka [F]dál, spí tam v [Gmi]růžích.:|
