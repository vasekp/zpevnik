\pisen{R. Müller}{Srdce jako kníže Rohan}

\sloka
[F]Měsíc je jak Zlatá bula [C]Sicilská,
[Ami]stvrzuje, se že kdo chce, ten se [G]dopíská,
[F]pod lampou jen krátce, v přítmí [C]dlouze zas,
[Ami]otevře ti Kobera a [G]můžeš mezi [F]nás.

\sloka
Moje teta, tvoje teta, parole,
dvaatřicet karet křepčí na stole,
měsíc svítí sám a chleba nežere,
ty to ale koukej trefit, frajere. Protože

\ref
[F]Dnes je valcha u starýho [C]Růžičky,
[Ami]dej si prachy do pořádny [G]ruličky.
[F]Co je na tom, že to není [C]extra nóbl byt?
[Ami]Srdce jako kníže Rohan [G]musíš mít.

\sloka
Ať si přes den docent nebo tunelář,
herold svatý pravdy nebo jinej lhář,
tady na to každej kašle zvysoka,
pravda je jen jedna – slova proroka říkaj že

\ref
Když je valcha u starýho Růžičky,
budou v celku nanic všechny řečičky.
Buďto trefa nebo kufr – smůla nebo šnit,
jen to srdce jako Rohan musíš mít.

\sloka
Kdo se bojí, má jen hnědý kaliko,
možná občas nebudeš mít na mlíko,
jistě ale poznáš co si vlastně zač,
svět nepatřil nikomu kdo nebyl hráč. A proto

\ref
Ať je valcha u starýho Růžičky
nebo pouť až k tváři Boží rodičky,
ať je válka, červen, mlha, bouřka nebo klid,
srdce jako kníže Rohan musíš mít.

\ref
Dnes je valcha u starýho Růžičky,
když si malej, tak si stoupni na špičky.
Malej nebo nachlapenej Cikán, Brňák, Žid,
srdce jako kníže Rohan musíš mít.

\ref
Dnes je valcha u starýho Růžičky,
dej si prachy do pořádny ruličky…

\ref
Dnes je valcha u starýho Růžičky,
dej si prachy do pořádny ruličky…

\ref
Ať je valcha u starýho Růžičky…
