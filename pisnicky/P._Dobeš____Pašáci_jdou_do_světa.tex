\pisen{P. Dobeš}{Pašáci jdou do světa}

\sloka
[D]Pašáci jdou do světa, [A7]guláš, pivo, trumpe[D]ta,
[A7]slunce stoupá oblohou, [D]zelenaj’ se volše,
[E7]kdekdo už je na nohou, ků[A7]rovec kůru kouše,
[D]nad hlavou nám letí včelí roj
a kdesi [A7]hluboko v nás šlape šicí [D]stroj.

\ref
[D]Před námi je hotel, [G]okna samý květ,
a [A7]hvězd, co může mít snad jenom [D]Interhotel Svět.
Spousty volných postelí a [G]jídla nízkých cen
a [A7]neznají tam zavírací [D]den.

\sloka
[G]Na Brněnské přehradě [D7]sešli jsme se v dobré nála[G]dě,
[D7]děvčata jdou po hrázi [G]jen v Evině rouše,
[A7]nikomu nic neschází, ků[D7]rovec kůru kouše,
[G]a ta jižní Morava
[D7]dobré vínko lidem podá[G]vá.

\sloka
A ti černí, to nejsou Arabi, to jsou ostravští parabi,
mají rádi Radegast, když jim v břiše šplouše,
rubou uhlí pro svou vlast, kůrovec kůru kouše,
Staříč, Paskov, Stonava,
Rudý říjen, Bezruč, Zárubek.

\zlom

\sloka
Řeka Váh je malý Jang’c’ťjang
a každé dievča už má vo výbave tank,
jedů chlapci do Tater dolů bez motoru,
Václav Klaus je amatér, kôrovec žiere kôru,
do videnia, do počutia, hoj, ahoj,
šlape přes nás šicí stroj.

\sloka
My jsme ti hradecký otroci a nemůžem si jinak pomoci,
až obejdem celý svět, oceány, souše,
vrátíme se domů, zpět, kde kůrovec kůru kouše,
kde se lidi strkaj’ v tramvaji,
a když se sejdou, tak si zpívají.

\ref
[C]Svět je balón, který letí [F]na ohřátý vzduch,
až [G7]přeletíme, zbude po nás [C]na obloze pruh.
Svět je balón, který s námi [F]časem odletí
do [G7]jednadvacátýho stole[C]tí.

\ref
[D]Před námi je hotel…
