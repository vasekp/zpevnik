\pisen{J. Čeřovská}{Řekni, kde ty kytky jsou}

\sloka
[C]Řekni, kde ty [Ami]kytky jsou, [F]co se s nima [G7]mohlo stát,
[C]řekni, kde ty [Ami]kytky jsou, [F]kde mohou [G7]být,
[C]dívky je tu [Ami]během dne [F]otrhaly [G7]do jedné,
[F]kdo to kdy [C]pochopí, [F]kdo to kdy [G7]pocho[C]pí?

\sloka
Řekni, kde ty dívky jsou, co se s nima mohlo stát,
řekni, kde ty dívky jsou, kde mohou být,
muži si je vyhlédli, s sebou domů odvedli,
kdo to kdy pochopí, kdo to kdy pochopí?

\sloka
Řekni, kde ti muži jsou, co se s nima mohlo stát,
řekni, kde ti muži jsou, kde mohou být,
muži v plné polní jdou, do války za zemi svou,
kdo to kdy pochopí, kdo to kdy pochopí?

\sloka
A kde jsou ti vojáci, co se s nima mohlo stát,
a kde jsou ti vojáci, kde mohou být,
řady hrobů v zákrytu, meluzína kvílí tu,
kdo to kdy pochopí, kdo to kdy pochopí?

\sloka
Řekni, kde ty hroby jsou, co se s nimi mohlo stát,
řekni, kde ty hroby jsou, kde mohou být.
Co tu kytek rozkvétá od jara do léta,
kdo to kdy pochopí, kdo to kdy pochopí.

\sloka
= 1.
