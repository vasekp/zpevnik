\pisen{K. Kryl}{Důchodce}

\sloka
[Ami]Kalhoty [C]roztřepené [Ami]na vnitřní [C]straně,
[Ami]zápěstí [C]roztřesené [Ami]a seschlé [C]dlaně,
kterými [E7]žmoulá kůrku [Ami]chleba sebranou z [E7]pultu v závod[Ami]ce,
tak tedy [E7]říkejme mu [Ami]třeba [E]důchodce.

\sloka
Rukávy u zimníku odřením lesklé,
čekává u rychlíků na zbytky ve skle,
stydlivě sbírá nedopalky ztracené v místech pro chodce
a pak je střádá do obálky důchodce.

\ref
[F]V neděli vysedává [Ami]na lavičce v sadech
a starou [G]špacírkou si podepírá [Ami]ustaranou hlavu,
[F]sluníčko vyhledává, [G]naříká si na dech
a vůbec [E]nikdo už mu neupírá právo na únavu.

\sloka
Na krku staré káro a úzkost v hlase,
když žebrá o cigáro, zlomí se v pase,
namísto díků sklopí zraky a není třeba žalobce,
\hvezda
když vidím [F]tyhle [G]lidský [Ami]vraky, třesu se [G]strachem před příz[Ami]raky,
že jednou [G]ze mne bude [E]taky [Ami]důchodce…
