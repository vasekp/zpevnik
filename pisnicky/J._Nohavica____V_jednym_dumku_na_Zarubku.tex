\pisen{J. Nohavica}{V jednym dumku na Zarubku}

\sloka
V [C]jednym dumku [G]na Zarubku [G7]mjal raz chlopek [C]švarnu robku
a ta robka [G]toho chlopka [G7]rada nemja[C]la.
Bo ten jeji chlopek [G]dobrotisko byl,
[G7]on tej svoji robce [C]všicko porobil,
čepani ji pomyl, [G]bravku daval žrať,
[G7]děcko musel koli[C]bať.

\sloka
Robil všecko, choval děcko, taky to byl dobrotisko,
ale robka toho chlopka rada nemjala.
Štvero novych šatu, štvero střevice,
do kostela nešla, enem k muzice,
same šminkovani, sama parada,
chlopka nemjala rada.

\sloka
A chlopisku – dobrotisku slze kanu po fusisku,
jak to vidi, jak to slyši, jaka robka je.
Dožralo to chlopka, že tak hlupy byl,
do hospody zašel, vyplatu přepil,
a jak domu přišel, řval, jak hrom by bil,
a tu svoju robku zbil.

\sloka
Včil ma robka rada chlopka, jak on piska, ona hopka,
Hanysku sem, Hanysku tam, ja ťa rada mam.
Věřte mi, luďkově, že to tak ma byť,
raz za čas třa robce kožuch vyprašiť,
pak je ona hodna tak jak ovečka
a ma rada chlopečka.
