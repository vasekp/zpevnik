\pisen{J. Nohavica}{Krajina po bitvě}

\sloka
[Ami]Míříme na sebe, míříme já a ty, [Emi]
[Ami]ticho je kolem, jen pes štěká za vra[C]ty,
[Emi]na louce leží mrtví [Ami]motýli,
[F]padá déšť [G]do es[E7]šálků,
[Dmi]jsme poslední dva, kteří [Ami]přežili
[E7]tuhletu dlouhou [Ami]válku, [Emi7 Ami Emi]hmm.

\sloka
Míříme na sebe, míříme já a ty,
v uších nám ještě zní vybuchlé granáty
a smrtka s kosou dělá resumé,
prochází se v bílé róbě,
no a my dva teď tady ležíme
v zákopech proti sobě.

\sloka
Míříme na sebe, míříme já a ty,
oba jsme utekli hrobníkům z lopaty,
ve stovkách velikánských útoků
štěstí nám oběma přálo
a teď nás dělí jenom sto kroků
a je to moc, či málo?

\sloka
Myslíme na sebe, myslíme já a ty,
co včera platilo, dneska už neplatí,
však ještě hrůza visí nad krajem,
těžké je mít se rádi,
když jsme si postříleli navzájem
své nejlepší kamarády.

\sloka
Nevíme o sobě, nevíme vůbec nic,
vzduch krví prosycen dýcháme z plných plic,
pach smrti pod kůží je zarytý,
končí se dějství prvé,
ještě nám tady zbývá k prolití
dvakrát šest litrů krve.

\sloka
Míříme na sebe, míříme já a ty,
ospalí, žízniví, hladoví, vousatí,
nebe se šeří, už se blíží noc,
oči jsou těžké jako kámen,
ach, koho žádat o radu a o pomoc,
když oba usínáme?

\sloka
A tak míříme na sebe, míříme já a ty,
padají hvězdy až obzor je hvězdnatý,
pod jedním nebem oba ležíme,
hřejivá je náruč matky Země,
a jak tak spíme, oba ve snu kráčíme:
já k tobě a ty ke mně.
