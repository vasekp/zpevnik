\pisen{W. Ryvola}{Tereza}

\sloka
Ten [C]den, co vítr [D]listí z města [G]svál,
můj [C]džíp se vracel, [D]jako by se [Emi]bál,
že [C]asfaltový [D]moře odliv [G]má
a [C]stáj že svýho [D]koně [E]nepozná.

\ref
Řekni, [G]kolik je na světě, kolik je takovejch [D]měst?
Řekni, [Ami]kdo by se vracel, když všude je tisíce [Emi]cest.
Tenkrát, [G]když jsi mi, Terezo, řekla, že ráda mě [D]máš,
tenkrát [Ami]ptal jsem se, Terezo, kolik mi polibků [Emi]dáš
napos[D]led, napos[G]led.

\sloka
Já z dálky viděl město v slunci stát
a dál jsem se jen s hrůzou musel ptát,
proč vítr mlátí spoustou okenic,
proč jsou v ulicích auta, jinak nic.

\ref
Řekni…

\sloka
Do prázdnejch beden zotvíranejch aut
zaznívá odněkud něžnej tón flaut
a v závějích starýho papíru
válej se černý klapky z klavírů.

\ref
Řekni…

\sloka
Tak loudám se tím hrozným městem sám
a vím, že Terezu už nepotkám.
Jen já tu zůstal s prázdnou ulicí
a osamělý město mlčící.

\ref
Řekni…
