\pisen{K. Kryl}{Tekuté písky}

\sloka
Jak je to [Ami]hezké, [E]když se někdo [Ami]žení,
jak je to [Dmi]smutné, když se někdo [G]vdává,
gumový [Dmi]kotouč bije do hra[E]zení,
chybějí [Ami]branky, [E]nikdo nepři[Ami]hrává.
V ruině [C]fasád, skrytých za le[G]šení,
zestárlí [G7]mladí za kvartýr a [C]stravu,
vlečem se [Dmi]časem, zpola udu[E]šeni
v tekutých [Ami]píscích [E]gubernijních [Ami]mravů.

\napis{M.}
{\sl (po každé sloce)} [*F G C Ami F E Ami E]

\sloka
Zabouchli dveře? Dereme se zadem,
zakleti v Knize lesů, vod a strání,
nařvaná tlama hlásá za výkladem,
že konec snů je koncem milování.
Gumový kotouč bije do hrazení,
není-li v kapse, nikdo nerozdává,
ne, není hezké, když se někdo žení,
ne, není smutné, když se někdo vdává.

\sloka
Být špatným hercem – nad to v světě není:
není-li dramat, stačí operetka,
gumový kotouč bije do hrazení,
nelze-li zkraje – tedy odprostředka.
Husita spílá obrněným vozům
a čas si žádá ledakterou hlavu,
zbudeš-li bez ní – k čemu je ti rozum
v tekutých píscích gubernijních mravů?

\sloka
Lešení skrývá paranoiu fasád,
žvanění tupců místo rozhovoru,
než v klidu zdechneš, nezbývá než nasát,
utopit zbytek odvahy a vzdoru.
Vyplníš jméno, místo narození
a sumu cifer, jež se nepřiznává,
tak už to bývá, když se někdo žení,
tak už to bývá, když se někdo vdává.

\sloka
Ze školy děti nesou vysvědčení,
zdalipak tuší, o čem se ti zdává,
když v noci žehlíš, stárnouc nad pečením?
O troše lásky, jíž se nedostává?
Nakoupíš chleba, vodu po holení
a zapřeš víru, jež se nevyznává,
tak už to bývá, když se někdo žení,
tak už to bývá, když se někdo vdává.

\sloka
Mlčení skrývá paranoiu žití,
mlčíš a zdobíš okna pro oslavu.
Víc nežli duše platí živobytí
v tekutých píscích gubernijních mravů.
Z bouřlivé vášně letmé pohlazení,
němota padá na zamrzlou vodu,
gumový kotouč bije do hrazení
a jen pár týdnů zbývá do rozvodu.
