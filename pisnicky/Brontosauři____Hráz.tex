\pisen{Brontosauři}{Hráz}

\sloka
[G]Stál tam na stráni [G\maj]dům, v něm židle a [G6]stůl,
pár kůží a [G]krb, co dřevo z něj [Ami]voní,
s [D]jarem, když máj rozdá barvy svý,
tu [D7]sosna krásná nad chajdou se [G]skloní
a [G]říčka, když stříbrným [G\maj]hávem se přikryje s [G6]ránem,
svý ahoj jí [G]dáš a pak je tu [E]den.
[C]Zazní údolím kytara tvá, [Emi]ta píseň ráno uvítá,
svět s [D7]ním, svět s [G]ním.

\sloka
Touláš se po lese, touláš a jenom tak bloumáš
a koruny stromů tě uvítaj rosou,
víš, že času je dost, to znáš,
a možná potkáš někde dívku bosou,
po slůvkách, který se říkaj, po dnech něžných stisků
vás uvítá chajda a zas je tu den.
Zazní údolím kytara tvá, ta píseň ráno uvítá,
svět s ním, svět s ním.

\sloka
Však náhle volání táhlé ti přeruší snění
a oznámí všem: je poslední den,
voda zaplaví údolí,
sosnu, chajdu, pohled zabolí.
Ta hráz je potřebná všem, však zabíjí den,
co nosil tě v náručí romantickém.
Zazní údolím bolest tvá, ta bolest ráno uvítá,
svět s ním, svět s ním.
