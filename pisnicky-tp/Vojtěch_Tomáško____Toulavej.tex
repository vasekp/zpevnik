% -*-coding: utf-8 -*-

\zp{Toulavej}{Vojtěch Tomáško}

\zs
Někdo <Ami>z vás, kdo chutnal <G>dálku, jeden z <Ami>těch, co rozu<E>měj,

ať vám <Ami>poví, proč mi <G>říkaj, proč mi <F>říkaj Toula<Ami>vej.
\ks

\zs
Kdo mě zná a v sále sedí, kdo si myslí, je mu hej,

tomu zpívá pro všední den, tomu zpívá Toulavej.
\ks

\zr
<F>Sobotní ráno mě <G>neuvidí <G7>u cesty s klukama <C>stát.

<F>Na půdě celta se <G>prachem stydí

<F>a starý songy jsem <G>zapomněl hrát, zapomněl <Ami>hrát.
\kr

\zs
Někdy v noci je mi smutno, často bývám doma zlej,

malá daň za vaše \uv{umí}, kterou splácí Toulavej.
\ks

\zs
Každej měsíc jiná štace, čekáš, kam tě uložej,

je to fajn, vždyť přece zpívá, třeba smutně, Toulavej.
\ks

\zr\kr

\zs
Vím, že jednou někdo přijde, tiše pískne: \uv{No tak jdem,}

známí kluci ruku stisknou, řeknou: \uv{Vítej, Toulavej.}
\ks

\zs
Budou hvězdy jako tenkrát, až tě v očích zabolej,

celou noc jim bude zpívat jeden blázen – Toulavej.
\ks

\zr
Sobotní ráno mi poletí vstříc, budeme u cesty stát

vypráším celtu a můžu vám říct,

že na starý songy si vzpomenu rád, vzpomenu rád.
\kr

\zs = 1. \ks

\kp
